% These lines tell TeXShop to typeset with xelatex, and to open 
% and save the source with Unicode encoding.

%!TEX TS-program = xelatex 
%!TEX encoding = UTF-8 Unicode

\documentclass[12pt]{article}

%%% La distribution XeteX permet d'utiliser facilement une fonte externe (ici Helvetica).
%%% On a besoin de xunicode (parce que les fichiers texte sont en unicode) et de fontspec
%%% pour que XeteX fonctionne bien.
\usepackage{xunicode}
\usepackage{fontspec}
\setromanfont[Mapping=tex-text]{Helvetica}


\usepackage{graphicx}
\usepackage[frenchb]{babel}

%%% liens et références dans les PDF
\usepackage{hyperref}
\hypersetup{
    unicode=true,          % non-Latin characters in Acrobat’s bookmarks
    pdftoolbar=true,        % show Acrobat’s toolbar?
    pdfmenubar=true,        % show Acrobat’s menu?
    pdffitwindow=false,     % window fit to page when opened
    pdfstartview={FitH},    % fits the width of the page to the window
    pdftitle={The Island},    % title
    pdfauthor={Peter Watts},     % author
    pdfsubject={Traduction en Français},   % subject of the document
    pdfcreator={Guillaume B},   % creator of the document 
    pdfkeywords={the island} {traduction} {Francais}, % list of keywords
    pdfnewwindow=true,      % links in new window
    colorlinks= true,       % false: boxed links; true: colored links
    linkcolor= blue,          % color of internal links
    citecolor= blue,        % color of links to bibliography
    filecolor= blue,      % color of file links
    urlcolor=blue           % color of external links
}

%%% Pas de numéro de page sur la première page des chapitres
\makeatletter
\let\ps@plain=\ps@empty
\makeatother


%%% Cul de lampe -> trois étoiles
\newlength{\cotetriangle}
\setlength{\cotetriangle}{2ex}
\newcommand*\troisetoile{%
  \begingroup % pour que le \centering soit local
                         % ainsi que les précautions de fontes qui suivent
  \normalsize  % taille habituelle
  \mdseries     % en graisse moyenne
  \upshape      % et en romain
  \par\nobreak
  \vspace*{1.5 cm}
  \centering
  *\hspace{\cotetriangle}*\hspace{\cotetriangle}*
  \par\penalty-150 % par exemple
  \vspace*{1.5 cm}
  \endgroup
}

%%% Commande de 'transit': comment on passe d'un chapitre à un autre
\newcommand{\transit}{
	\troisetoile
%%%	\newpage
}

%%% pas d'indentation au début des paragraphes
\parindent=0pt
%\setlength{\parskip}{0pt}  
\setlength{\parskip}{2ex plus 0.5ex minus 0.2ex}

%%% plus petites marges
\addtolength{\textwidth}{2cm}
\addtolength{\oddsidemargin}{-1cm}


\title{L'Île}
\author{Peter Watts}
%\date{Dimanche, 24 Octobre 2010}  %%% à décommenter un jour?
 \date{}

\begin{document} 

%%% les pages du début n'ont pas de numéro de page
\pagestyle{empty}
\maketitle
 \cleardoublepage

 
\input{../header}
\cleardoublepage

%%% le reste a un numéro de page, on commence à 1
\pagestyle{myheadings}
\markright{\bf{\emph{The Island}}\hfill}
\setcounter{page}{1}
 
%%% le texte en lui-même
\input{../Chapitres/Intro}
\transit
\input{../Chapitres/Chapter01}
\transit
Lorsque j'arrive sur le pont, il est seul devant les écrans, les yeux pleins d'icônes et de trajectoires. Je vois peut être un peu de moi dans ses gestes.

\og Je n'ai pas saisi ton prénom \fg , demandais-je, même si je l'avais déjà trouvé dans le manifeste. À peine a-t-on été présenté que je suis déjà en train de lui mentir.

\og Dixon \fg, répond-t-il, tout en gardant ses yeux sur l'hologramme.

Il a plus de dix mille ans. En vie pendant peut être une vingtaine. Je me demande ce qu'il sait, qui il a rencontré pendant ces décades éparpillées: est-ce qu'il connait Ismael, ou Connie? Est-ce qu'il sait si Sanchez a finalement réussi à passer outre son problème avec l'immortalité?

Je me demande tout ça, mais je ne lui pose pas la question. Il y a des règles.

Je regarde autour de moi. \og Il n’y a que nous comme équipe?\fg

Dix acquiesce. \og Pour le moment. On ramènera plus de monde si besoin. Mais\ldots\fg{} Sa voix s'éteint.

\og Oui? \fg

\og Rien. \fg

Je le rejoins à l'hologramme. Des forme semi-transparentes sont suspendues comme une fumée gelée, avec des codes-couleur. Nous sommes sur le bord d'un nuage de poussière mo\-lé\-cu\-laire. Chaud, semi-or\-ga\-nique, plein de ma\-té\-riaux de base: for\-mal\-dé\-hyde, éthylène-glycol, les pré\-biotes usu\-els. Un bon emplacement pour une construction rapide. Une naine rouge brille au centre de l'holo. Chimp l'a nommée DHF428, pour des raisons dont j'ai oublié de prêter attention depuis longtemps.

\og Alors, mets-moi au courant \fg , dis-je.

Son regard est impatient, irrité même. \og Toi aussi? \fg

\og Qu'est ce que tu veux dire? \fg

\og Comme les autres. Durant les dernières constructions. Chimp pouvait simplement leur envoyer les données, mais eux voulaient \emph{parler} tout le temps.\fg

Merde, son lien est encore actif. Il est \emph{connecté}.

J'ai un sourire forcé. \og C'est juste une\ldots{} une tradition culturelle, j'imagine. On parle de beaucoup de choses, ça nous aide à\ldots{} nous reconnecter. Après avoir été débranché si longtemps."

\og Mais c'est \emph{lent}\fg , se plaint-t-il.

Il ne sait pas. Pourquoi ne sait-il pas?

\og On a encore une demi année-lumière \fg , lui fais-je remarquer. \og Il y a urgence?\fg

Le coin de sa bouche tressaille. \og Les Neumanns sont partis à l'heure.\fg{} Pour confirmer, un nuage de points violets clignote dans l'hologramme, cinq trillion de kilomètres devant nous. \og Encore en train d'aspirer de la poussière pour la plupart, mais ils ont été chanceux en tombant sur deux gros astéroïdes, et les raffineries ont été finies en avance. Premiers composants déjà extrudés. Mais Chimp a vu ces fluctuations dans l'apport de lumière — principalement infrarouge, mais qui s'étend au visible.\fg{} L'hologramme commence à clignoter: la naine rouge passe en accéléré.

Effectivement, ça \emph{clignote}.

\og Pas aléatoire, je suppose.\fg

Dix incline sa tête un peu sur le côté, presque pour acquiescer. 

\og Trace un graphe temporel.\fg{} Je n'ai jamais pu m'empêcher de hausser le ton, juste un peu, quand je m'adresse à Chimp. Obéissante (\emph{Obéissante}. C'en est presque comique), l'IA efface le panorama spatial pour le remplacer par ceci:

\noindent\includegraphics[width=\textwidth]{../dots}

\og Séquence répétitive \fg , me dit Dix. \og Les bips ne changent pas, mais l'es\-pa\-ce\-ment croît de manière log-linéaire, boucle toutes les 92.5 secondes. Chaque cycle démarre à 13.2 bips/s, et diminue dans le temps.\fg

\og Aucune chance que ça soit naturel? Un trou noir qui oscillerait au centre de l'étoile, peut-être?\fg

Dix secoue la tête, ou quelque chose qui y ressemble: un plongeon diagonal du menton qui, quelque part, semble négatif. \og Mais c’est beaucoup trop simple pour contenir de l'information. Pas comme une vraie conversation. Plus comme\ldots{} eh bien, un cri. \fg

Il n’a que partiellement raison. Il y a peut être très peu d'informations, mais c’est bien assez. \emph{Nous sommes ici. Nous sommes intelligents. Nous sommes assez puissants pour faire d'une étoile un simple variateur de lumière}.

Peut être pas un si bon emplacement pour construire, finalement.

Je serre les lèvres. \og L'étoile nous salue. C'est ce que tu penses.\fg

\og Peut être. Salue \emph{quelqu'un} en tout cas. Mais c’est trop simple pour être une pierre de Rosette. Ce n'est pas une archive, ça n'est pas compressé. Non plus une séquence de Bonferroni ou Fibonacci, ni Pi. Pas même une table de multiplication. Rien qui puisse poser les bases d'un dialecte.\fg

Et pourtant. Un signal intelligent.

\og On a besoin de plus d'infos\fg , explique Dix, se proclamant malgré lui maitre de l'aveu\-glément évident.

J'acquiesce. \og Les Neumanns.\fg

\og Euh, qu'est-ce qu'ils ont?\fg

\og On met en place une grille. On utilise un tas d'yeux passables pour en faire un bon. Ce serait plus rapide que de fabriquer une grille de ce côté ou de re-configurer une des fabriques déjà sur-site.\fg

Ses yeux s'agrandissent. Un instant il semble presque effrayé pour quel\-que raison. Mais l'instant passe et il refait cet étrange mouvement de tête. \og Ça dépouillerait trop de ressources réservées à la construction, non?\fg

\og Ce serait le cas\fg , confirme Chimp.

Je refoule un reniflement. \og Si tu es aussi inquiet pour tes performances sur cette construction, Chimp, introduis donc le risque potentiel posé par une intelligence assez puissante pour contrôler le flux énergétique d'un soleil entier.\fg

\og Je ne peux pas\fg , admet-il. \og Je n'ai pas assez d'informations.\fg

\og Tu n’en a \emph{aucune}. À propos d'une chose qui pourrait probablement stopper net cette mission si elle voulait. Alors on devrait peut-être essayer d'en acquérir.\fg

\og D'accord. Neumanns réassignés.\fg

La confirmation est visible sur l'un des panneaux, une séquence complexe d’instru\-ctions de pas de danse lancées dans le vide spatial. Dans six mois, une centaine de robots auto-réplicants se re-posi\-tionne\-ront en une grille de surveillance temporaire; et quatre mois plus tard, nous aurons peut-être plus que le vide spatial comme données sur lesquelles débattre.

Dix me regarde comme si j'avais prononcé une formule magique.

\og Il a beau commander ce vaisseau,\fg{} lui dis-je, \og il est tout de même vraiment stupide. Des fois il faut simplement lui expliquer les choses à haute voix.\fg

Il semble vaguement offensé, mais il est clairement surpris. Il ne savait pas cela. Il ne le \emph{savait pas}.

Qui donc l'a élevé durant tout ce temps? Qui doit répondre de lui?

Pas moi.

\og Appelez moi dans dix mois\fg , leur dis-je. \og Je retourne au lit.\fg
\transit
 \input{../Chapitres/Chapter03}
\transit
 \input{../Chapitres/Chapter04}
\transit
 \input{../Chapitres/Chapter05}
\transit
 Je tourne au bout du couloir et heurte Dix, debout immobile comme un golem au milieu de mon salon. Je fais un saut de un mètre en l'air.

"\emph{Qu'est ce que tu fous ici?}"

Il semble surpris par ma réaction. "Je voulais —parler," dit-il après un moment.

"On n'entre \emph{jamais} chez quelqu'un sans y être invité!"

Il recule d'un pas, bégayant: "Je voulais, je voulais—"

"Parler. Et tu fais ça en public. Sur le pont, ou dans les communs, ou —si c'est vraiment pour ça, tu peux tout simplement m'\emph{appeler}."

Il hésite. "Tu disais que— tu \emph{voulais} discuter face à face. Tu disais que c’était une \emph{tradition culturelle}."

C'est vrai. Mais pas \emph{ici}. Ici c'est chez \emph{moi}, ce sont mes \emph{quartiers privés}. L'absence de verrou à la porte fait partie du protocole de sécurité, ce n'est pas une invitation à entrer chez moi et s'\emph{installer} pour m'attendre, à rester debout comme une satanée \emph{décoration}.

Je grogne: "Pourquoi est tu \emph{réveillé}, d'ailleurs? On n'est pas supposé être debout avant deux mois."

"J'ai demandé à Chimp de me réveiller en même temps que toi."

Cette putain de machine.

"Et pourquoi \emph{toi} tu es réveillée?" me demande-t-il, restant planté là.

Je soupire, capitulant, et je m'effondre dans le premier siège à ma portée. "Je voulais juste faire une passe sur les données préliminaires." Le mot "\emph{seule}", implicite, devrait être évident.

"Et alors?"

Manifestement, il ne l'est pas. Je décide de jouer le jeu pour un moment. "Il semblerait qu'on parle à une— une île. À peu près six mille kilomètres de diamètre. Pour la partie pensante, en tout cas. La membrane qui l'en\-toure est pratiquement vide. Enfin je veux dire, cette chose est \emph{vivante}. L'in\-té\-gra\-lité est photosynthétique, ou quelque chose dans le genre. Ça mange, je pense. Pas certaine de quoi."

"Le nuage moléculaire," me dit Dixon. "Les composés organiques sont partout. En plus ça concentre la matière à l'intérieur de la sphère."

Je hausse les épaules. "Le truc, c'est qu'il y a une limite à la taille d'un cerveau mais cette chose est \emph{énorme}. C'est\ldots"

"Improbable," murmure-t-il, presque pour lui-même.

Je me tourne pour l'observer; le siège se reformant autour de moi. "Qu'est ce que tu veux dire?"

"L'Île fait vingt huit millions de kilomètres carré? La sphère entière fait sept quintillion. Et l'Île se trouvant justement entre nous et 428, ça fait— une chance sur cinquante mille milliards."

"Continue."

Il ne peut pas. "Euh, c'est juste\ldots~ juste \emph{improbable}."

Je ferme les yeux. "Comment peux-tu être assez fort pour calculer tout ça dans ta tête en un clin d'oeil, et être assez stupide pour louper la conclusion évidente?"

De nouveau ce regard paniqué, abattu. "Ne pense p— Je ne suis pas—"

"C'est impossible. C'est \emph{astronomiquement} impossible que nous soyons justement alignés sur la seule tâche intelligente sur une sphère d'une unité astronomique et demi de diamètre. Ce qui signifie\ldots"

Il ne dit rien. La perplexité de son regard en est presque moqueuse. J'ai envie de le frapper.

Mais finalement, l'ampoule s'allume: "Il y a, euh, plus qu'une île? Oh! \emph{Plein} d'îles!"

Dire que cette créature fait partie de l'équipage. Ma vie dépendra certainement de lui un jour. C'est une pensée effrayante.

J'essaye de l'éloigner pour le moment. "Il y a probablement une population entière de ces choses, saupoudrée sur la membrane comme— comme des kystes, je suppose. Chimp ne sait pas combien, mais on n'en capte qu'un seul pour le moment, alors ils doivent être assez espacés."

Il fronce les sourcils d'une façon différente maintenant. 

"Pourquoi \emph{Chimp}?"

"Qu'est ce que tu veux dire?"

"Pourquoi l'appeler Chimp?"

"On l'appelle Chimp." Parce que la première chose à faire pour humaniser quelque chose, c'est de lui donner un nom.

"J'ai cherché. C’est un raccourci pour \emph{chimpanzé}. Un animal stupide."

"En fait, je crois que les Chimpanzés étaient supposés être assez intelligents."

"Pas comme nous. Ils ne pouvaient même pas \emph{parler}. Chimp peut parler. Il est \emph{bien plus} intelligent que ces animaux. Ce nom —c'est une insulte."

"Qu'est ce que ça peut te faire?"

Il me regarde, simplement.

J'écarte les bras. "Okay, c'est pas un chimpanzé. On l'appelle comme ça parce qu'il a à peu près le même nombre de synapses."

"Donc vous lui donnez un petit cerveau, et ensuite vous vous plaignez qu'il soit stupide à longueur de journée."

Ma patience est pratiquement épuisée. "Est-ce que tu as quelque chose à dire, ou bien tu expires juste du CO$_{2}$ pour—"

"Pourquoi ne pas le faire plus intelligent?"

"Parce qu'on ne peut jamais prédire le comportement d'un système plus compliqué que nous. Et si tu veux qu'un projet reste sur les rails après que tu sois parti, tu ne laisses pas les rênes à quelque chose qui va développer ses propres motivations." Bordel de merde, je pensais que quelqu'un lui avait parlé de la loi d'Ashby.

"Donc ils l'ont lobotomisé," commente Dix après un moment.

"Non. Ils ne l'ont pas rendu stupide, ils l'ont \emph{construit} comme cela."

"Peut être plus intelligent que tu ne le crois. Puisque tu es si intelligente, puisque tu as tes motivations propres, pourquoi est-\emph{il} encore aux commandes?"

"Ne te jette pas des fleurs", dis-je.

"Quoi?"

Je laisse échapper un sinistre sourire. "Tu ne fais que suivre les ordres d'un tas de systèmes \emph{bien plus} compliqués que tu ne l'es." On peut leur baisser notre chapeau, aussi; morts depuis des millénaires, et ces satanés instigateurs du projet tirent \emph{encore} les ficelles. 

"Non je ne— \emph{Je} suis des ordres?—"

"Je suis désolée mon cher." Je souris amicalement à ma descendance idiote. "Je ne parlais pas à toi. Je parlais à la chose qui fait sortir tous ces sons de ta bouche."

Dix devient plus blanc que mes culottes.

Je laisse tomber mes faux-semblants. "Qu'est ce que tu croyais, Chimp? Que tu pouvais envoyer cette marionnette envahir ma vie sans que je m'en aperçoive?"

"Pas— Ce n'est pas— c'est \emph{moi}," bégaie Dix. "C’est \emph{moi} qui parle."

"Il te \emph{souffle} les mots. Sais-tu seulement ce que 'lobotomisé' \emph{signifie}?" Je secoue la tête, dégoûtée. "Tu crois que j'ai oublié comment fonctionne l'interface juste parce qu'on a tous brûlé la nôtre?" Une caricature de surprise commence à se dessiner sur sa figure. "Oh, n'\emph{essaye} même pas. Tu as déjà vécu d'autres constructions, il est impossible que tu ne le saches pas. Et tu sais aussi que l'on a fermé nos liens intérieurs. Et il n'y a rien que notre seigneur et maître puisse faire à ce propos parce qu'il a \emph{besoin} de nous, et nous avons donc atteint ce que tu pourrais appeler un \emph{arrangement}."

Je ne crie pas. Mon ton est glacial, mais ma voix n'est qu'un écho. Et pourtant, Dix semble se \emph{plier} en deux.

Je réalise qu'il y a une opportunité à cet instant.

Je dégèle un peu ma voix. Je parle gentiment: "Tu peux le faire aussi, tu sais. Brûler ton lien. Je te laisserais même revenir ici ensuite, si tu le veux encore. Juste pour— parler. Mais pas avec cette chose dans ta tête."

La panique prend place sur ses traits, et contre toute attente, ça me brise le coeur. "Je \emph{peux pas}," plaide-t-il. "Grâce à ça j'\emph{apprends}, grâce à ça je m'\emph{entraîne}. La \emph{mission}\ldots"

Je ne sais pas lequel est en train de parler, donc je réponds aux deux: "Il y a plus d'une façon de procéder à la mission. On a plus de temps qu'il n'en faut pour les essayer toutes. Dix sera le bienvenu ici quand il sera tout seul."

Ils font un pas dans ma direction. Un autre. Une main, tressautante, monte de leur côté gauche, comme pour saisir quelque chose, et il y a quelque chose sur cette figure de travers que je n'arrive pas à reconnaître.

"Mais je suis ton \emph{fils}," disent-ils.

Je ne daigne même pas nier cela devant eux.

"Sortez de mes quartiers."
\transit
 \input{../Chapitres/Chapter07}
\transit
 \input{../Chapitres/Chapter08}
\transit
 \input{../Chapitres/Chapter09}
\transit
 \input{../Chapitres/Chapter10}
\transit
 \input{../Chapitres/Chapter11}
\transit
 \input{../Chapitres/Chapter12}
\transit
 \input{../Chapitres/Chapter13}
\transit
 \input{../Chapitres/Chapter14}
\transit
 \input{../Chapitres/Chapter15}
\transit
 \input{../Chapitres/Chapter16}
\transit

\cleardoublepage

\pagestyle{empty}

\input{../Notes}
 \input{../footer}

 
\end{document}
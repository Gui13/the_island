Je tourne au bout du couloir et heurte Dix, debout immobile comme un golem au milieu de mon salon. Je fais un saut de un mètre en l'air.

"\emph{Qu'est ce que tu fous ici?}"

Il semble surpris par ma réaction. "Je voulais —parler," dit-il après un moment.

"On n'entre \emph{jamais} chez quelqu'un sans y être invité!"

Il recule d'un pas, bégayant: "Je voulais, je voulais—"

"Parler. Et tu fais ça en public. Sur le pont, ou dans les communs, ou —si c'est vraiment pour ça, tu peux tout simplement m'\emph{appeler}."

Il hésite. "Tu disais que— tu \emph{voulais} discuter face à face. Tu disais que c’était une \emph{tradition culturelle}."

C'est vrai. Mais pas \emph{ici}. Ici c'est chez \emph{moi}, ce sont mes \emph{quartiers privés}. L'absence de verrou à la porte fait partie du protocole de sécurité, ce n'est pas une invitation à entrer chez moi et s'\emph{installer} pour m'attendre, à rester debout comme une satanée \emph{décoration}.

Je grogne: "Pourquoi est tu \emph{réveillé}, d'ailleurs? On n'est pas supposé être debout avant deux mois."

"J'ai demandé à Chimp de me réveiller en même temps que toi."

Cette putain de machine.

"Et pourquoi \emph{toi} tu es réveillée?" me demande-t-il, restant planté là.

Je soupire, capitulant, et je m'effondre dans le premier siège à ma portée. "Je voulais juste faire une passe sur les données préliminaires." Le mot "\emph{seule}", implicite, devrait être évident.

"Et alors?"

Manifestement, il ne l'est pas. Je décide de jouer le jeu pour un moment. "Il semblerait qu'on parle à une— une île. À peu près six mille kilomètres de diamètre. Pour la partie pensante, en tout cas. La membrane qui l'en\-toure est pratiquement vide. Enfin je veux dire, cette chose est \emph{vivante}. L'in\-té\-gra\-lité est photosynthétique, ou quelque chose dans le genre. Ça mange, je pense. Pas certaine de quoi."

"Le nuage moléculaire," me dit Dixon. "Les composés organiques sont partout. En plus ça concentre la matière à l'intérieur de la sphère."

Je hausse les épaules. "Le truc, c'est qu'il y a une limite à la taille d'un cerveau mais cette chose est \emph{énorme}. C'est\ldots"

"Improbable," murmure-t-il, presque pour lui-même.

Je me tourne pour l'observer; le siège se reformant autour de moi. "Qu'est ce que tu veux dire?"

"L'Île fait vingt huit millions de kilomètres carré? La sphère entière fait sept quintillion. Et l'Île se trouvant justement entre nous et 428, ça fait— une chance sur cinquante mille milliards."

"Continue."

Il ne peut pas. "Euh, c'est juste\ldots~ juste \emph{improbable}."

Je ferme les yeux. "Comment peux-tu être assez fort pour calculer tout ça dans ta tête en un clin d'oeil, et être assez stupide pour louper la conclusion évidente?"

De nouveau ce regard paniqué, abattu. "Ne pense p— Je ne suis pas—"

"C'est impossible. C'est \emph{astronomiquement} impossible que nous soyons justement alignés sur la seule tâche intelligente sur une sphère d'une unité astronomique et demi de diamètre. Ce qui signifie\ldots"

Il ne dit rien. La perplexité de son regard en est presque moqueuse. J'ai envie de le frapper.

Mais finalement, l'ampoule s'allume: "Il y a, euh, plus qu'une île? Oh! \emph{Plein} d'îles!"

Dire que cette créature fait partie de l'équipage. Ma vie dépendra certainement de lui un jour. C'est une pensée effrayante.

J'essaye de l'éloigner pour le moment. "Il y a probablement une population entière de ces choses, saupoudrée sur la membrane comme— comme des kystes, je suppose. Chimp ne sait pas combien, mais on n'en capte qu'un seul pour le moment, alors ils doivent être assez espacés."

Il fronce les sourcils d'une façon différente maintenant. 

"Pourquoi \emph{Chimp}?"

"Qu'est ce que tu veux dire?"

"Pourquoi l'appeler Chimp?"

"On l'appelle Chimp." Parce que la première chose à faire pour humaniser quelque chose, c'est de lui donner un nom.

"J'ai cherché. C’est un raccourci pour \emph{chimpanzé}. Un animal stupide."

"En fait, je crois que les Chimpanzés étaient supposés être assez intelligents."

"Pas comme nous. Ils ne pouvaient même pas \emph{parler}. Chimp peut parler. Il est \emph{bien plus} intelligent que ces animaux. Ce nom —c'est une insulte."

"Qu'est ce que ça peut te faire?"

Il me regarde, simplement.

J'écarte les bras. "Okay, c'est pas un chimpanzé. On l'appelle comme ça parce qu'il a à peu près le même nombre de synapses."

"Donc vous lui donnez un petit cerveau, et ensuite vous vous plaignez qu'il soit stupide à longueur de journée."

Ma patience est pratiquement épuisée. "Est-ce que tu as quelque chose à dire, ou bien tu expires juste du CO$_{2}$ pour—"

"Pourquoi ne pas le faire plus intelligent?"

"Parce qu'on ne peut jamais prédire le comportement d'un système plus compliqué que nous. Et si tu veux qu'un projet reste sur les rails après que tu sois parti, tu ne laisses pas les rênes à quelque chose qui va développer ses propres motivations." Bordel de merde, je pensais que quelqu'un lui avait parlé de la loi d'Ashby.

"Donc ils l'ont lobotomisé," commente Dix après un moment.

"Non. Ils ne l'ont pas rendu stupide, ils l'ont \emph{construit} comme cela."

"Peut être plus intelligent que tu ne le crois. Puisque tu es si intelligente, puisque tu as tes motivations propres, pourquoi est-\emph{il} encore aux commandes?"

"Ne te jette pas des fleurs", dis-je.

"Quoi?"

Je laisse échapper un sinistre sourire. "Tu ne fais que suivre les ordres d'un tas de systèmes \emph{bien plus} compliqués que tu ne l'es." On peut leur baisser notre chapeau, aussi; morts depuis des millénaires, et ces satanés instigateurs du projet tirent \emph{encore} les ficelles. 

"Non je ne— \emph{Je} suis des ordres?—"

"Je suis désolée mon cher." Je souris amicalement à ma descendance idiote. "Je ne parlais pas à toi. Je parlais à la chose qui fait sortir tous ces sons de ta bouche."

Dix devient plus blanc que mes culottes.

Je laisse tomber mes faux-semblants. "Qu'est ce que tu croyais, Chimp? Que tu pouvais envoyer cette marionnette envahir ma vie sans que je m'en aperçoive?"

"Pas— Ce n'est pas— c'est \emph{moi}," bégaie Dix. "C’est \emph{moi} qui parle."

"Il te \emph{souffle} les mots. Sais-tu seulement ce que 'lobotomisé' \emph{signifie}?" Je secoue la tête, dégoûtée. "Tu crois que j'ai oublié comment fonctionne l'interface juste parce qu'on a tous brûlé la nôtre?" Une caricature de surprise commence à se dessiner sur sa figure. "Oh, n'\emph{essaye} même pas. Tu as déjà vécu d'autres constructions, il est impossible que tu ne le saches pas. Et tu sais aussi que l'on a fermé nos liens intérieurs. Et il n'y a rien que notre seigneur et maître puisse faire à ce propos parce qu'il a \emph{besoin} de nous, et nous avons donc atteint ce que tu pourrais appeler un \emph{arrangement}."

Je ne crie pas. Mon ton est glacial, mais ma voix n'est qu'un écho. Et pourtant, Dix semble se \emph{plier} en deux.

Je réalise qu'il y a une opportunité à cet instant.

Je dégèle un peu ma voix. Je parle gentiment: "Tu peux le faire aussi, tu sais. Brûler ton lien. Je te laisserais même revenir ici ensuite, si tu le veux encore. Juste pour— parler. Mais pas avec cette chose dans ta tête."

La panique prend place sur ses traits, et contre toute attente, ça me brise le coeur. "Je \emph{peux pas}," plaide-t-il. "Grâce à ça j'\emph{apprends}, grâce à ça je m'\emph{entraîne}. La \emph{mission}\ldots"

Je ne sais pas lequel est en train de parler, donc je réponds aux deux: "Il y a plus d'une façon de procéder à la mission. On a plus de temps qu'il n'en faut pour les essayer toutes. Dix sera le bienvenu ici quand il sera tout seul."

Ils font un pas dans ma direction. Un autre. Une main, tressautante, monte de leur côté gauche, comme pour saisir quelque chose, et il y a quelque chose sur cette figure de travers que je n'arrive pas à reconnaître.

"Mais je suis ton \emph{fils}," disent-ils.

Je ne daigne même pas nier cela devant eux.

"Sortez de mes quartiers."
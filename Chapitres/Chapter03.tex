C'est comme s'il n'était jamais parti. Je remonte sur le pont et il est là, absorbé par les écrans. DHF428 occupe tout l'hologramme, une grosse orbe rouge, transformant les traits de mon fils en un masque de démon.

Il m'accorde un rapide coup d'oeil, les yeux ronds, ses doigts ani\-més com\-me s'ils étaient électrifiés. "Les Neumanns ne le voient pas."

Je suis encore un peu vaseuse du dégel. "Voir qu—"

"La \emph{séquence}!" Sa voix est au bord de la panique. Il oscille d'avant en arrière, répartissant son poids d'un pied à l'autre.

"Montre moi."

Les écrans se séparent en deux. Deux naines identiques brulent de\-vant moi maintenant, chacune peut-être deux fois la taille de mon poing. À gauche, la vue depuis \emph{Éri}: DHF428 bégaie comme elle le faisait auparavant, comme elle l'a probablement fait durant les dix derniers mois. Sur la droite, une vue composée: une grille d'interférences construite par une myriade de Neumanns espacés précisément, leurs yeux rudimentaires empilés et parallaxés donnant quelque chose qui approche la haute définition. Le contraste, des deux côtés, a été adapté pour souligner les clins d'oeil de la naine à un oeil humain.

À part que ça ne cligne que du côté gauche de l'écran. Sur la droite, 428 brille normalement, telle une simple bougie.

"Chimp, y a-t-il une chance pour que la grille ne soit pas assez sensible pour voir les fluctuations?"

"Non."

"Ah." J'essaye de penser à une raison qui le ferait mentir à ce propos.

"Ça n'a aucun \emph{sens}," se plaint mon fils.

"Ça en a," murmurai-je, "si ce n'est pas l'étoile qui scintille."

"Mais elle scintille—" Il fait claquer sa langue. "On peut la voir cli— attends, tu veux dire quelque chose \emph{derrière} les Neumanns? Entre— entre eux et nous?"

"Mmmm."

"Une sorte de \emph{filtre}." Dix se relaxe un peu. "Est-ce qu'on ne l'aurait pas déjà vu? Les Neumanns l'auraient heurté sur leur route?"

Je remet ma voix en mode CommChimp. "Quel est le champ de vision du télescope d'\emph{Éri}?"

"Dix-huit minutes d’arc," m'apprend Chimp. "Au niveau de 428, le cône fait trois point trente quatre secondes-lumières de large."

"Agrandis à cent secondes-lumière."

La moitié d’image provenant d'\emph{Éri} s'agrandit alors, oblitérant le point de vue dissident. Un instant, le soleil remplit de nouveau l'intégralité de l'ho\-lo\-gram\-me, peignant le pont d'une couleur cramoisi. Puis il fond, comme dévoré de l'intérieur.

Je remarque un peu de flou dans l'image. "Est-ce que tu peux nettoyer le bruit?"

"Ce n'est pas du bruit," m'apprend Chimp. "Ce sont des poussières et du gaz moléculaire."

Je cligne des yeux. "Quelle densité?"

"Estimée à cent mille atomes par mètre cube."

Deux ordres de magnitude trop haut, même pour une nébuleuse. "Pourquoi si lourd?" Nous aurions certainement détecté n'importe quel puits de gravité assez fort pour maintenir autant de matière dans le voisinage.

"Je ne sais pas," me répond Chimp.

J'ai la mauvaise impression que moi, si. "Étend le champ à cinq cent se\-con\-des-lumière. Accentue en fausses couleurs le proche infrarouge."

L'espace grandit dans l'hologramme, inquiétant. Le petit soleil à son cen\-tre, la taille d'un ongle, brille de plus en plus: une perle incandescente dans une eau boueuse.

"Mille secondes-lumière," ordonnais-je.

"Là," murmure Dix: l'espace réel réapparait sur les bords de l'holo\-gram\-me, sombre, clair, immaculé. 428 repose au coeur d'un linceul laiteux de forme sphérique. On trouve ces choses, parfois, des débris d'étoiles compagnon dont les convulsions ont balancé du gaz et des radiations à des années lumières. Mais 428 n'est pas un reste de nova. C'est une \emph{naine rouge}, placide, au milieu de sa vie. Quelconque.

Excepté le fait qu'elle est située au centre d'une sphère de gaz ténue de 1,4 unités astronomiques de diamètre. Et le fait que cette bulle ne se \emph{diffuse} pas, ne \emph{s'atténue} pas, ni ne disparaît progressivement dans cette nuit accueillante. Non, à moins qu'il y ait quelque chose de complètement faux dans cette représentation, cette petite nébuleuse sphérique s'étend environ à 350 secondes-lumière de son astre et \emph{s'arrête} net, ses frontières bien plus abruptes que ce que la nature autorise.

Pour la première fois depuis des millénaires, je n'accroche pas mon insert cortical. Cela me demande une éternité pour taper sur le clavier dans ma tête, pour découvrir les réponses que je connais déjà.

Les chiffres arrivent. "Chimp, je veux que tu accentues les raies à 335, 500 et 800 nanomètres."

Le linceul autour de 428 s'éclaire alors comme les ailes d'une libellule, comme une bulle de savon.

"C'est \emph{magnifique}," murmure mon fils, abasourdi.

"C'est photosynthétique," lui dis-je.